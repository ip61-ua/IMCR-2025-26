\documentclass[11pt]{beamer}
\usetheme{Madrid}

\usepackage[spanish]{babel}
\usepackage[utf8]{inputenc}
\usepackage[T1]{fontenc}
\usepackage{lmodern}
\usepackage{graphicx}
\usepackage{booktabs}
\usepackage{xcolor}
\usepackage{hyperref}
\usepackage{csquotes}
\usepackage{booktabs}

\title[Aula: Peligros y pizarra]{\textit{State of the Art}: peligros y pizarra del aula inteligente}
\subtitle{Ingeniería de Mantenimiento de Computadores y Redes}
\author{Propuesta por Ivan}
\date{\today}

\setcounter{tocdepth}{2}

\AtBeginSection[]
{
  \begin{frame}{Índice}
    \tableofcontents[currentsection, hideothersubsections]
  \end{frame}
}

\AtBeginSubsection[]
{
  \begin{frame}{Índice}
    \tableofcontents[currentsection, currentsubsection, subsectionstyle=show/shaded/hide]
  \end{frame}
}

\begin{document}

\begin{frame}
  \titlepage
\end{frame}
  
\begin{frame}{Índice}
  \tableofcontents
\end{frame}

\section{Detección de peligros}

\begin{frame}{Cambio de paradigma}
  
  \begin{block}{Objetivo}
    Complementar sistemas reactivos basados en \textit{hardware} dedicado con sistemas proactivos basados en \textbf{visión artificial} (en adelante, CV por \textit{Computer Vision}).
  \end{block}
    
  \vspace{0.5cm}
    
  \begin{itemize}
  \item \textbf{\textit{Hardware} tradicional:} Sensores unifuncionales (humo, humedad, PIR). Entradas activas: botón, palanca, lógica con enfoque más \enquote{síncrono}.
  \item \textbf{Enfoque \textit{software}/IA:} Uso de cámaras CCTV (cámaras de circuito cerrado, usadas para monitoreo de vídeo de seguridad) existentes + procesamiento en el borde (\textit{Edge Computing}) o \enquote{nube}.
  \end{itemize}
\end{frame}


\subsection{Riesgos en el aula}
\begin{frame}{¿Qué riesgos hay en un aula?}
  \begin{itemize}
  \item \textbf{Seguridad:}
    \begin{itemize}
    \item Violencia física (peleas, acoso).
    \item Presencia de armas o bultos de objetos contundentes.
    \item Intrusos en horarios restringidos.
    \end{itemize}
  \item \textbf{Integridad del entorno:}
    \begin{itemize}
    \item Incendios (análisis espectral y de movimiento).
    \item Inundaciones (análisis de textura en suelos).
    \item Bloqueo de salidas de emergencia (análisis espacial).
    \item Terremotos y actividad sísmica (temblores).
    \item Desprendimientos de objetos así como lanzamiento.
    \item Vandalismo (pintadas, comportamiento canalla...).
    \end{itemize}
  \item \textbf{Salud y bienestar:}
    \begin{itemize}
    \item Caídas y desmayos.
    \item Detección de pánico o aglomeraciones.
    \item Contaminación acústica (ruido estridente, chillidos...).
    \end{itemize}
  \end{itemize}
\end{frame}

\subsection{Tecnología útil}
\begin{frame}{\large{\textbf{YOLO}} (\textit{You Only Look Once})}
  \begin{itemize}
  \item \textbf{Rápido y preciso}.
  \item Sensibilidad a oclusiones.
  \item \textbf{Navaja suiza}: Detección, segmentación, seguimiento, poses...
  \item Código abierto (OSS, \textit{Open Source Software}). \url{https://github.com/ultralytics/ultralytics}.
  \item Usado también para \textbf{atención de alumnos}: \url{https://www.mdpi.com/1424-8220/25/22/6972}
  \item En \textbf{tiempo real} a través de cámaras IP a servidor local.
  \item Usado para detectar robo, violencia, vandalismo, fuego...
  \item Plausible en \textit{hardware} limitado si se relajan restricciones.
  \end{itemize}
\end{frame}

\begin{frame}{\textbf{Google MediaPipe}}
  \begin{itemize}
  \item Rápida sturación con multitud.
  \item Eficiente al poder ejecutarse en \textbf{móviles} (pensado para sin GPU dedicada).
  \item Mejor captura de la profundidad.
  \item OSS.
  \item \textbf{Navaja suiza}.
  \item ¿Optimizado para Rasberry Pi? (\enquote{\textit{¿Gemini hablando bien de Google?}}).
  \item \url{https://github.com/google-ai-edge/mediapipe}.
  \end{itemize}
\end{frame}

\begin{frame}{\textbf{CNN} (\textit{Convolutional Neural Networks})}
  \begin{itemize}
  \item \textbf{Extracción de características:} Aprende automáticamente a detectar bordes, texturas y formas sin programación manual.
  \item \textbf{Jerarquía Visual:} 
    $$ \text{Bordes} \to \text{Formas simples} \to \text{Objetos complejos} $$
  \item \textbf{Invarianza:} Reconoce el objeto independientemente de su posición $(x,y)$ en la imagen.
  \item El \textbf{\textit{kernel}} (filtro) recorre la matriz de píxeles calculando el producto escalar para detectar patrones.
  \end{itemize}
\end{frame}

\begin{frame}{Violencia}
  \begin{itemize}
  \item La violencia es un \textbf{acto temporal} $\rightarrow$ secuencia memoria.
  \item \textbf{YOLO} por si solo le cuesta $\rightarrow$ YOLO-Pose. Precisamos de una TPU si hacer en el \textit{edge}.
  \item \textbf{YOLO-Pose/Google MediaPipe}+\textbf{LSTM} ({\small \textit{Long Short Term Memory}}).
    \begin{itemize}
    \item Clasificador temporal.
    \item \textit{Hardware} modesto $\rightarrow$ \textbf{privacidad}.
    \item \textbf{Oclusiones y OSS}.
    \end{itemize}
  \item \textbf{SlowFast}
    \begin{itemize}
    \item Preciso. De Facebook. OSS.
    \item Computación pesada.
    \item Dos modos de funcionamiento: \textit{slow} (¿quién es?), \textit{fast} (movimiento)
    \end{itemize}
  \item Ejemplo detección de violencia y vandalismo \url{https://github.com/Ab-code00/SurakshaAI---Real-Time-Suspicious-Activity-Detection-System}. 
  \end{itemize}
\end{frame}

\begin{frame}{Detección de intrusos}
  \begin{itemize}
  \item \textbf{Objetivo}: saltar alarma si viene alguien no deseado en determinado horario. 
  \item Cuádruple reto: \textbf{reconocimiento facial} + \textbf{horarios} + \textbf{lista blanca} + \textbf{privacidad}.
  \item Variantes:
    \begin{enumerate}
    \item Naive: hay personas $\land$ fuera de horario $\rightarrow$ alarma.
    \item Reconocer caras $\rightarrow$ ¿Privacidad? ¿Niños? ¿Ejercer derecho?. Necesidad de una DB actualizada $\rightarrow$ lío legal. 
    \item ¿Quién decide entrar y hasta dónde puede? (Absurdo no dejar entrar visita, familiares, servicios públicos, nuevo personal $\notin$ DB) $\rightarrow$ ¿1984?
    \end{enumerate}
  \item Merodeos: \url{https://github.com/nwojke/deep_sort}.
  \item \url{https://github.com/hectorpadin1/Network-Intrusion-Detection-System}.
  \item Otro enfoque, analizar la red: \url{https://yardenfalik.github.io/IDS-Project/}. (Esto sale en la teoría de Sistemas Distribuidos).
  \end{itemize}
\end{frame}

\begin{frame}{Incendios e inundaciones}
  Fuego
  \begin{itemize}
  \item \textbf{no tiene forma fija, \enquote{parpadea}, brilla, reflja}.
  \item Puede producirse en \textbf{sitios ocultos} donde la cámara no vea.
  \item Versiones \textit{custom} de YOLO.
  \end{itemize}
  {\large $\implies$ Detección por IA como complementario para anticipar.}
  \normalsize
  
  \vspace{.5cm}
  Agua
  \begin{itemize}
  \item \textbf{No se busca encontrar líquido} sino otras características.
  \item \textbf{Reflejos} (da igual si es sucia o transparente).
  \item Detección del suelo mojado píxel a píxel (U-Net, DeepLabV3+).
  \item Causas naturales $\rightarrow$ otros asuntos, meteo...
  \end{itemize}
  \large $\implies$ Interés en prevenir (tuberías rotas, fraturas...). Conocer  aula concreta.
  
\end{frame}

\begin{frame}{Caídas y desmayos}
  \begin{itemize}
  \item Heurísticas sobre la pose del esqueleto.
    \begin{enumerate}
    \item Cambio en la \textbf{relación de aspecto}.
    \item \textbf{Velocidad} de descenso (¿se sienta o se cae?).
    \item \textbf{Inactividad} (¿tropiezo o desmayo?).
    \end{enumerate}
  \item Otras situaciones.
    \begin{itemize}
    \item Deporte $\rightarrow$ \textit{fine tunning}, más movimiento.
    \item \textit{Trolling} $\rightarrow$ al menos, 18 añitos tiene la criatura.
    \end{itemize}
  \item ¿Oclusión? Inferencia en base a lo visible.
  \item Tecologías a gastar: YOLO, YOLO-pose, BoT-SORT y ByteTrack.
  \item \textbf{OpenPifPaf}: \url{https://github.com/cwlroda/falldetection_openpifpaf}.
  \item \textbf{AlphaPose}: \url{https://github.com/GajuuzZ/Human-Falling-Detect-Tracks}.
  \item \textbf{Multicam}: \url{https://github.com/taufeeque9/HumanFallDetection}.
  \end{itemize}
\end{frame}

\begin{frame}{Bloqueo de salidas}
  Detectar objetos (mochilas, mesas) que permanecen en vías de evacuación o que impidan una salida normal. \textbf{Diferenciar un bloqueo real} de un tránsito momentáneo.
  
  \begin{block}{Algoritmo: IoU (\textit{Intersection over Union}) Temporal}
    Sea $P_{salida}$ el polígono de la salida, $B_{bulto}$ la caja del objeto y $T$ el tiempo transcurrido desde que se detecta la obstrucción:
    $$ \text{Solape} = \frac{\text{Área}(P_{salida} \cap B_{bulto})}{\text{Área}(P_{salida})} $$
    $$ (\text{Solape} > 0.3) \land (T > 60s) \rightarrow \text{Alerta} $$
  \end{block}
\end{frame}



\begin{frame}{Audio: gritos, auxilio, ruido estridente...}
  \begin{itemize}
  \item \textbf{Clasificador de sonido de Tensorflow} YAMNet \url{https://www.tensorflow.org/hub/tutorials/yamnet?hl=es-419}.
  \item Micrófono I2S: \url{https://opencircuit.es/producto/fermion-i2s-mems-microphone-breakout}.
  \end{itemize}
  
  \begin{columns}
    \column{0.5\textwidth}
    \textbf{Procesamiento (YAMNet):}
    \begin{enumerate}
    \item \textbf{Entrada:} Micrófono I2S (MEMS) en la Raspberry Pi.
    \item \textbf{Preprocesar:} Transformada de Fourier $\to$ Mel-Spectrogram (Redes de los Computadores).
    \item \textbf{Clasificación:} CNN ligera (MobileNet) entrenada en \textit{AudioSet}.
    \end{enumerate}
    
    \column{0.4840829572\textwidth}
    \centering
    \begin{block}{Eventos Críticos}
      \begin{itemize}
      \item \textbf{Gritos:} Agresión o Pánico.
      \item \textbf{Destrucción:} Vandalismo.
      \item \textbf{Palabras clave:} \enquote{¡Socorro!}, \enquote{¡Ayuda!} (TinyML).
      \end{itemize}
    \end{block}
    
    Ejemplo: \url{https://github.com/Varun-310/SCREAM}
  \end{columns}
  
  \vspace{0.3cm}
  \textbf{Privacidad garantizada.} Almacenar ¿qué eventos peligrosos (\textit{Edge}).
  \end{frame}



\begin{frame}{No todo se detecta mejor con IA}
  \begin{itemize}
  \item \textbf{Fuegos, inundaciones}.
  \item \textbf{Sismos} (ESP32 + MPU-6050): \url{https://github.com/serdaraltin/earthquake-warning-system}.
  \end{itemize}
\end{frame}

\begin{frame}{Demostraciones}
  \begin{itemize}
  \item \textbf{Fuego}: \url{https://github.com/MuhammadMoinFaisal/FireDetectionYOLOv8}.
  \item \textbf{Armas}: \url{https://www.nature.com/articles/s41598-025-07782-0}, \url{https://github.com/swatified/Weapon-Detection-System}.
  \item \textbf{Violencia}: \url{https://www.youtube.com/watch?v=a2xWqkFDYuU}, \url{https://www.youtube.com/watch?v=D4mjEBgAXPU}, \url{https://www.youtube.com/watch?v=qeFrjFa5Rxc} 
  \item \textbf{Caídas}: \url{https://www.youtube.com/watch?v=vEtsmg7-fWs}  
  \item \textbf{Ataque de pánico, aglomeraciones}: \url{https://github.com/moego0/panic-attack-detector}, \url{https://github.com/jinay-k-jain/Panic_detector_CCTV}.
  \item \textbf{Vandalismo}: \url{https://ieeexplore.ieee.org/document/10895139}.
  \end{itemize}
\end{frame}

\subsection{Infraestructura}
\begin{frame}{¿Qué tenemos a disposición?}
  \begin{itemize}
  \item 600€ de presupuesto en nuevo \textit{hardware}.
  \item Electrónica de otros años.
    \begin{itemize}
    \item Placas de Arduino.
    \item Rasberry Pi.
    \item ESP32.
    \end{itemize}
  \item Ordenadores del aula.
    \begin{itemize}
    \item GPU: Nvidia GTX 1660.
    \end{itemize}
  \end{itemize}

  \begin{center}
    \LARGE{\textbf{¡Relajar restricciones!}}
  \end{center}

  \begin{enumerate}
  \item Una cámara $\rightarrow$ un propósito.
  \item Reducir FPS $\rightarrow$ no tiempo real.
  \item Procesamiento centralizado (SPOF). ¿cómo vigilancia en urbanización?
  \item Combiando estrategias (o aplicando divide y vencerás para el reparto de responsabilidad): \textit{edge} detecta que hay personas, \textit{server} averigua quienes son.
  \end{enumerate}
\end{frame}

\begin{frame}{Ejecución en \textit{hardware} limitado}
  \small
    No todo el \textit{hardware} puede procesar las millones de operaciones matriciales de una CNN. Clasificación por niveles:
    
    \vspace{0.3cm}
    \begin{table}[]
      \centering
      \resizebox{\textwidth}{!}{
        \begin{tabular}{l l l l}
          \toprule
          \textbf{Dispositivo} & \textbf{Capacidad CNN} & \textbf{Tecnología} & \textbf{Uso} \\
          \midrule
          \textbf{Raspberry Pi 5} & \textbf{Alta} & Python, TensorFlow, PyTorch & Corre YOLO/MediaPipe para análisis complejo. \\
          \midrule
          \textbf{ESP32-S3} & \textbf{Limitada (TinyML)} & TFLite Micro, C++ & Sensor auxiliar. Solo clasificación simple (Person/No-Person). \\
          \midrule
          \textbf{Arduino Uno} & \textbf{Nula} & - & Solo control de relés/luces. No procesa imagen. \\
          \midrule
          \textbf{Arduino Portenta} & \textbf{Media} & OpenMV, Edge Impulse & Similar al ESP32 pero grado industrial. \\
          \bottomrule
        \end{tabular}
        }
      \end{table}
      \vspace{0.2cm}
    Es importante la \textbf{cuantización.} Para correr en ESP32, los modelos se comprimen de \texttt{float32} a \texttt{int8}, reduciendo el tamaño del modelo de 50MB a $<500$KB.

    \begin{center}
      \LARGE{\textbf{¡Optimizar!}}
    \end{center}
  \end{frame}

\section{Digitalización del Aula: Pizarra Inteligente}
\begin{frame}{Captura Inteligente de Contenido}
El sistema utiliza una cámara cenital o frontal para digitalizar la pizarra analógica automáticamente.
    
    \begin{columns}
        \column{0.5\textwidth}
        \begin{block}{Disparadores (Triggers)}
            La captura se activa mediante lógica de software:
            \begin{itemize}
                \item \textbf{Botón Físico:} Ad-hoc por el profesor al terminar una explicación.
                \item \textbf{App Estudiante:} Petición individual vía WebSocket.
                \item \textbf{Modo "Ad Quorum":} Se dispara solo si el $X\%$ de los alumnos lo solicita en 30 segundos.
            \end{itemize}
        \end{block}
        
        \column{0.5\textwidth}
        \begin{alertblock}{Detección Automática}
            Mediante \textbf{Background Subtraction}, el software detecta cuando la pizarra está "llena" y el profesor se aparta, disparando la foto automáticamente.
        \end{alertblock}
    \end{columns}
\end{frame}

\begin{frame}{Privacidad y Post-Procesamiento (GDPR)}
    Uno de los retos del software en aulas es la privacidad de los alumnos y el profesor.
    
    \vspace{0.5cm}
    
    \textbf{Pipeline de Anonimización:}
    \begin{enumerate}
        \item \textbf{Detección de Rostros:} Uso de modelos ligeros (ej. MTCNN o Haar Cascades) para localizar caras en la imagen de la pizarra.
        \item \textbf{Aplicación de Blur (Difuminado):} Aplicación de filtro Gaussiano sobre las regiones de interés (ROIs) detectadas.
        \item \textbf{Rectificación de Perspectiva:} Algoritmo de homografía para transformar la foto lateral en una imagen plana "escaneada".
        \item \textbf{Mejora de Contraste:} Binarización adaptativa para convertir trazos de tiza/rotulador en digital nítido.
    \end{enumerate}
\end{frame}

% -----------------------------------------------------------------------------
% Conclusiones
% -----------------------------------------------------------------------------
\section{Conclusiones}

\begin{frame}{Resumen de la Propuesta}
    \begin{itemize}
        \item La infraestructura se basa en \textbf{software} sobre hardware genérico (cámaras), reduciendo costes de instalación.
        \item La IA permite una detección de peligros más rápida y contextual (ve el fuego, no solo huele el humo).
        \item La digitalización de la pizarra respeta la privacidad (blurring) y democratiza el apunte (quorum).
    \end{itemize}
    
    \vspace{1cm}
    \centering
    \textit{¿Preguntas?}
\end{frame}

\end{document}
