\documentclass[11pt]{beamer}
\usetheme{Madrid}

\usepackage[spanish]{babel}
\usepackage[utf8]{inputenc}
\usepackage[T1]{fontenc}
\usepackage{lmodern}
\usepackage{graphicx}
\usepackage{booktabs}
\usepackage{xcolor}
\usepackage{hyperref}
\usepackage{csquotes}
\usepackage{booktabs}

\title[Aula: Peligros y pizarra]{Tratamiento de peligros y pizarra en un aula inteligente}
\subtitle{Ingeniería de Mantenimiento de Computadores y Redes}
\author{Propuesta por Ivan}
\date{\today}

\setcounter{tocdepth}{2}

\AtBeginSection[]
{
  \begin{frame}
    \tableofcontents[currentsection, hideothersubsections]
  \end{frame}
}

\AtBeginSubsection[]
{
  \begin{frame}
    \tableofcontents[currentsection, currentsubsection, subsectionstyle=show/shaded/hide]
  \end{frame}
}

\begin{document}

\begin{frame}
  \titlepage
\end{frame}
  
\begin{frame}{Índice}
  \tableofcontents
\end{frame}

\section{Detección de peligros}

\begin{frame}{Cambio de paradigma}
  
  \begin{block}{Objetivo}
    Complementar sistemas reactivos basados en \textit{hardware} dedicado con sistemas proactivos basados en \textbf{Visión Artificial} (en adelante, CV por \textit{Computer Vision}).
  \end{block}
    
  \vspace{0.5cm}
    
  \begin{itemize}
  \item \textbf{\textit{Hardware} Tradicional:} Sensores unifuncionales (humo, humedad, PIR). Entradas activas: botón, palanca, lógica con enfoque más \enquote{síncrono}.
  \item \textbf{Enfoque \textit{Software}/IA:} Uso de cámaras CCTV (cámaras de circuito cerrado, usadas para monitoreo de vídeo de seguridad) existentes + procesamiento en el borde (\textit{Edge Computing}) o \enquote{nube}.
  \end{itemize}
\end{frame}

\begin{frame}{¿Qué riesgos hay en un aula?}
  \begin{itemize}
  \item \textbf{Seguridad:}
    \begin{itemize}
    \item Violencia física (peleas, acoso).
    \item Presencia de armas o bultos de objetos contundentes.
    \item Intrusos en horarios restringidos.
    \item Presencia de animales salvajes y peligrosos.
    \end{itemize}
  \item \textbf{Integridad del entorno:}
    \begin{itemize}
    \item Incendios (análisis espectral y de movimiento).
    \item Inundaciones (análisis de textura en suelos).
    \item Bloqueo de salidas de emergencia (análisis espacial).
    \item Terremotos y actividad sísmica (temblores).
    \item Desprendimientos de objetos así como lanzamiento.
    \item Vandalismo (pintadas, comportamiento canalla...).
    \end{itemize}
  \item \textbf{Salud y bienestar:}
    \begin{itemize}
    \item Caídas y desmayos.
    \item Detección de pánico o aglomeraciones.
    \item Contaminación acústica (ruido estridente, chillidos...).
    \end{itemize}
  \end{itemize}
\end{frame}

\subsection{Violencia}
\begin{frame}{Detectar la violencia}
  \large{\textbf{Hybrid CNN Xception + LSTM}}
  \begin{itemize}    
  \item LSTM = \textit{Long Short-Term Memory}.
  \item \textbf{Muy preciso} (98\% tasa de acierto) con \textit{datasets} clásicos.
  \item Eliminación de elementos redundantes.
  \item Requiere de una \textbf{GPU potente} (Nvidia RTX 3060+).
  \item Computación muy pesada.
  \item Masiva etiquetación de \textit{datasets} para mayor precisión.
  \item Propietaria (sujeta a licencias y costes).
  \item \textit{Paper} académico: \url{https://iajit.org/upload/files/Hybrid-CNN-Xception-and-Long-Short-Term-Memory-Model-for-the-Detection-of-Interpersonal-Violence-in-Videos.pdf}.
  \end{itemize}

  $\implies$ emplear \textbf{CNN} (Sistemas Inteligentes).
\end{frame}

\begin{frame}{Detertar la violencia}
  \large{\textbf{YOLO}}
  \begin{itemize}
  \item Código abierto (OSS, \textit{Open Source Software}).
  \item Usado también para \textbf{atención de alumnos}: https://www.mdpi.com/1424-8220/25/22/6972
  \end{itemize}
\end{frame}

\begin{frame}
  \url{https://www.youtube.com/watch?v=a2xWqkFDYuU}
  \url{https://www.youtube.com/watch?v=D4mjEBgAXPU}
  \url{https://www.youtube.com/watch?v=qeFrjFa5Rxc}
  \url{https://www.youtube.com/watch?v=vEtsmg7-fWs}

\end{frame}

\begin{frame}{¿Qué riesgos hay un aula?}
  \small
  \begin{table}[]
    \centering
    \begin{tabular}{p{2cm} p{4cm} p{6cm}}
      \toprule
      \textbf{Peligro} & \textbf{Detección Tradicional} & \textbf{Detección por Software (IA)} \\
      \midrule
      \textbf{Incendios} & Detectores de humo/térmicos. & \textbf{CNNs} entrenadas para reconocer patrones de flamas y comportamiento volumétrico del humo antes de que alcance el techo. \\
      \midrule
            \textbf{Amenazas \newline (Terrorismo)} & Botones de pánico, detectores de metales en puertas. & \textbf{Object Detection (YOLO/SSD)} para identificar armas. Análisis de pose (OpenPose) para detectar lenguaje corporal agresivo. \\
            \midrule
            \textbf{Inundaciones} & Sensores de humedad en zócalos. & Segmentación semántica para detectar cambios en la textura del suelo y acumulación de agua. \\
            \bottomrule
        \end{tabular}
    \end{table}
\end{frame}

\begin{frame}{Profundización Técnica: Algoritmos de Detección}
    Para la detección de amenazas mediante cámaras, se propone un pipeline de software:
    
    \vspace{0.3cm}
    
    \begin{enumerate}
        \item \textbf{Pre-procesamiento:} Mejora de imagen low-light (para aulas oscuras).
        \item \textbf{Inferencia:}
            \begin{itemize}
                \item \textit{Incendios:} Redes siamesas para diferenciar fuego real de pantallas/reflejos.
                \item \textit{Intrusión:} Reconocimiento facial (con listas de control) y re-identificación de personas.
            \end{itemize}
        \item \textbf{Alerta:} Integración API con servicios de emergencia (sin intervención humana si la certeza $> 95\%$).
    \end{enumerate}
\end{frame}

% =============================================================================
% SECCIÓN 2: UTILIDADES DE PIZARRA
% =============================================================================
\section{Digitalización del Aula: Pizarra Inteligente}

\begin{frame}{Captura Inteligente de Contenido}
    El sistema utiliza una cámara cenital o frontal para digitalizar la pizarra analógica automáticamente.
    
    \begin{columns}
        \column{0.5\textwidth}
        \begin{block}{Disparadores (Triggers)}
            La captura se activa mediante lógica de software:
            \begin{itemize}
                \item \textbf{Botón Físico:} Ad-hoc por el profesor al terminar una explicación.
                \item \textbf{App Estudiante:} Petición individual vía WebSocket.
                \item \textbf{Modo "Ad Quorum":} Se dispara solo si el $X\%$ de los alumnos lo solicita en 30 segundos.
            \end{itemize}
        \end{block}
        
        \column{0.5\textwidth}
        \begin{alertblock}{Detección Automática}
            Mediante \textbf{Background Subtraction}, el software detecta cuando la pizarra está "llena" y el profesor se aparta, disparando la foto automáticamente.
        \end{alertblock}
    \end{columns}
\end{frame}

\begin{frame}{Privacidad y Post-Procesamiento (GDPR)}
    Uno de los retos del software en aulas es la privacidad de los alumnos y el profesor.
    
    \vspace{0.5cm}
    
    \textbf{Pipeline de Anonimización:}
    \begin{enumerate}
        \item \textbf{Detección de Rostros:} Uso de modelos ligeros (ej. MTCNN o Haar Cascades) para localizar caras en la imagen de la pizarra.
        \item \textbf{Aplicación de Blur (Difuminado):} Aplicación de filtro Gaussiano sobre las regiones de interés (ROIs) detectadas.
        \item \textbf{Rectificación de Perspectiva:} Algoritmo de homografía para transformar la foto lateral en una imagen plana "escaneada".
        \item \textbf{Mejora de Contraste:} Binarización adaptativa para convertir trazos de tiza/rotulador en digital nítido.
    \end{enumerate}
\end{frame}

% -----------------------------------------------------------------------------
% Conclusiones
% -----------------------------------------------------------------------------
\section{Conclusiones}

\begin{frame}{Resumen de la Propuesta}
    \begin{itemize}
        \item La infraestructura se basa en \textbf{software} sobre hardware genérico (cámaras), reduciendo costes de instalación.
        \item La IA permite una detección de peligros más rápida y contextual (ve el fuego, no solo huele el humo).
        \item La digitalización de la pizarra respeta la privacidad (blurring) y democratiza el apunte (quorum).
    \end{itemize}
    
    \vspace{1cm}
    \centering
    \textit{¿Preguntas?}
\end{frame}

\end{document}
