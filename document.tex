\documentclass[]{article}

\usepackage{csquotes}
\usepackage{geometry}
\usepackage{hyperref}

\geometry{a4paper, top=2.5cm, bottom=2.5cm, left=2cm, right=2cm}
\title{Ingeniería de Mantenimiento de los Computadores y Redes}
\author{}

\begin{document}
	
	\begin{titlepage}
	\centering
	\vspace{0.5cm}
	{\Large \textbf{Universidad de Alicante}}\par
	{\large Escuela Politécnica Superior}\par
	
	\vspace{1.5cm}
	{\Large Grado en Ingeniería Informática}\par
	
	\vspace{1.5cm}
	{\huge \bfseries Ingeniería de Mantenimiento de los Computadores y Redes}\par
	
	\vspace{2cm}
	{\LARGE \bfseries Entrega 1}\par
	
	\vspace{0.5cm}
	{\huge \textit{Propuesta y análisis de requerimientos del proyecto}}\par
	
	\vspace*{\fill}
	
	\begin{flushright}
		\large
		\textbf{Autor}\\
		Ivan Parkhomchyk Patapchyk \par
		Samuel Alan Hoad \par
		Sergio Pascual \par
                Juan Antonio Bonillo \par
		\vspace{1.5cm}
		\textbf{Curso Académico}\\
		2025-2026\par
	\end{flushright}
\end{titlepage}

	
	\section{Definición de las funcionalidades}

        El grupo de entrada y salida (I/O, en adelante) tiene como objetivo primordial el de proveer una capa de abstracción del \textit{hardware}. Simplificando y definiendo protocolos comunes a la la comunicación.

        \begin{itemize}
          
        \item \textbf{Interfaz del \textit{hardware}}. Encrgados de transformar la entrada en una salida bien formateada y estándar para el resto de elementos. Por ejemplo: transformar la entrada analógica en un número con dos decimales como máximo.

        \item \textbf{Gestión de los actuadores}. Existe la obligación de llevar a cabo órdenes de alto nivel al que abstraigamos. Por ejemplo: que la orden \enquote{cerrar puerta} haga efectiva la acción del cierre físico de la puerta. (Centrándonos en la consecuencia).

        \item \textbf{Sistema de notificaciones}. Es indispensable la retroalimentación al usuario o de una orden. Con esto se busca dar un \textit{feedback} de cierta actividad eventualmente ocurrida en el acto. Estas se efectuarían mediante componentes físicos como pantallas, leds, dispositivos encargados emisión de sonido, zumbidos y otras señales.

        \item \textbf{Resiliencia a fallos}. El equipo debe garantizar el funcionamiendo adecuado frente a circunstancias adversas tales como la no disposición de un componente del que existe cierta relación e implicación entre ambos involucrados. 
          
        \end{itemize}

        \section{Necesidades materiales}
        \label{sec:nesmat}

        A lo largo de esta sección, quedan listadas los materiales que inicialmente hay planteados utilizar. Acompañas de un correspondiente justificación.

        \begin{itemize}

        \item Menester de \textbf{microcontroladores} como Arduino o RasberryPi que tengan conectividad inalámbrica.

          \subitem Esta última característica es fundamental por la posibilidad de proveer una comunicación libre de cableados de red. De esta forma es fácil la ubicación de algunos componentes.

          \subitem Estos tipos de micrcontroladores tienen una gran presencia en el mundo la automatización, entusiastas y educación. Por lo que no será de tanta dificultad encontrar componentes y documentación de calidad al estar sustentados por una amplia comunidad de usuarios.
          
        \item Para el funcionamiento independiente de una red WiFi ya existente y emplear una tecnología que viene como anillo al dedo es utilizar \textbf{LoRaWAN}.

          \subitem Diseñadas para dispositivos IoT de bajo consumo.

          \subitem Fácil instalación. Adicionalmente, abarca áreas de gran alcance con lo que la comunicación entre componentes alocados en la lejanía entre sí pudiesen operar con relativa facilidad.

        \item Los \textbf{relés de estado sólido} a diferencia de los mecánicos tradicionales generan menos ruido acústico (ideal para no molestar durante explicaciones o exámenes), tienen mayor vida útil (menor coste de mantenimiento)...

        \item El empleo de \textbf{servomotores} quedaría para la automatización la interacción con persianas, ventanas y puertas.

          \subitem Dependendiendo del objeto físico a tratar, se precisará de uno de alto torque para pesos pesados o uno más ajustados a las necesidades con lo que vaya a mover.

          \subitem Es gran interés establecer mecanismos para la detección de obstrucción y definir procedimientos de actuación antes tales condiciones de situación. Además, todo momento debería permitirse el uso de un modo manual para casos de necesidad concreta no contemplada o de urgencia.


        \item Cualquier espacio que precise de un acceso restringido (como cajones, armarios, taquillas y puertas) necesitará una \textbf{cerradurra}.

          \subitem En cualquier caso, será necesario contemplar los casos de emergencia en los cuales necesariamente deban relajarse o aumentarse esta serie de restricciones. Por ejemplo: deshabilitar el control de acceso si hay un incendio o mentener cerrado un cajón si alguien intenta manipular y sustraer materiales de un cajón.

          \subitem En función de cada caso, convendrá el uso cerraduras eletromagnéticas (\textit{maglocks}) o pestillos electrónicos (Selenoides).

        \item Implementación de \textbf{cámaras} proporciona grandes volúmenes de información que debe ser manejado con cierta delicadez dada la recopilación de datos. Existen diferentes tipos que pueden ser considerados en función de la utilidad que vayan a desempeñar.

          \subitem Las cámaras de baja resolución pueden servir para la lectura de QRs, códigos, tareas simples que no requiran de mucha precisión (movimiendo, conteo del personal). Estas cámaras suponen un coste económico por su precio y consumo.

          \subitem Las térmicas proveen de imágenes en las que reflejan matrices de calor, ayudarían a detectar sobrecalentamiento en un aula y a evaluar las condiciones climáticas del aula. Son una garantía de privacidad al captar imágenes del medio.
          \subitem Las cámaras de alta calidad pueden ser usadas a transmitir imágenes que, cuyo interés sea el análisis de estas propiamente. Los fotogramas por esta categoría de cámaras deben ser tratados con especial cuidado. Algunos usos frecuentes a este tipo fotoaparatos consisten en el reconocimiento de personas, ubicar pertenencias perdidas a sus respectivos dueños, precisión en la contabilidad del personal...

        \item Esta Universidad cuenta con un conjunto de servicios relacionados con el IoT (véase \url{https://smart.ua.es/}). Algunas funcionalidades más básicas están destinadas a la medición del tiempo y de la afluencia y ocupación. Por lo que sería de vigoroso interés poder establecer contacto para conseguir más información acerca del proyecto. En cualquier caso sería posible recrear una \enquote{estación meteorológica} por medio de la interconexión de los siguientes componentes para una mayor independencia de terceros: 

          \subitem Veletas para computar la velocidad del viento. Además de obtener la dirección del mismo.

          \subitem Pluviómetros para averiguar la cantidad de lluvia un preciso instante.

          \subitem Barómetros con tal de medir la presión atmosférica.

          \subitem Termómetros con el objetivo de conseguir la temperatura.

          \subitem Sensores de humedad.

        \item Para ofrecer una \textbf{mayor accesibilidad} en cualquier término: 

          \subitem La inclusión de líneas braile garantizan que las personas ciegas puedan enterner intrucciones con el tacto.

          \subitem El uso de la megafonía para alertar e informar de situaciones puede resultar en una mejor experiencias para el público en general.

          \subitem Los lectores NFC pueden facilitar el uso en general motivando así la utilización de tarjetas de contacto. En este aspecto es fundamental dar la garantía de ofrecer una seguridad frente a ataques de falsificación e identidad. Otra ventaja de esta clase de tarjetas es la familiaridad y facilidad de uso para todo tipo de público.
          
        \end{itemize}
        
\end{document}
